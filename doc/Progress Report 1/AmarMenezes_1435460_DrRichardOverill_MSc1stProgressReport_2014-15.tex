\documentclass{llncs}

\begin{document}
\begin{center}
	
	% Upper part of the page. The '~' is needed because \\
	% only works if a paragraph has started.
	
	\textsc{\LARGE King's College London\\ \small School of Natural and Mathematical Sciences\\ \small Department of Informatics
	}\\[1.5cm]
	
	\textsc{\Large MSc 1st Progress Report}\\[0.5cm]
	
	% Title
	\hrule~\\[0.4cm]
	{ \huge \bfseries Smartphone Malware Analysis \\[0.4cm] }
	\hrule~\\[1.5cm]
	
	% Author and supervisor
	\noindent
	\begin{minipage}[t]{0.4\textwidth}
		\begin{flushleft} \large
			\emph{Author:}\\
			Amar Menezes\\(1435460)
		\end{flushleft}
	\end{minipage}%
	\begin{minipage}[t]{0.4\textwidth}
		\begin{flushright} \large
			\emph{Supervisor:} \\
			Dr.~Richard E. Overill
		\end{flushright}
	\end{minipage}
	
	\vfill
	
	% Bottom of the page
	{\large \today}
\end{center}


\section{Summary}
Tasks Completed:
\begin{itemize}
	\item{Understanding the security models for Android, iOS and Windows Phone.}
	\item{Survey of attack vectors on the three platforms.}
	\item{Survey of approaches to analyse Android malware and analysis evasion techniques by malware. Tools used for manual and automated analysis.}
	\item{Initial draft of a malware analysis process.}
\end{itemize}
In Progress:
\begin{itemize}
	\item{A Survey on existing malware variants for Android documenting their origin (if known), mode of distribution, vulnerabilities being exploited, and its capabilities.}
	\item{Gathering tools for reversing apk files and statically analysing decompiled code.}
	\item{Gathering tools to build a sandbox for dynamic analysis.}
\end{itemize}
To be started:
\begin{itemize}
	\item{A survey of existing malware variants for iOS and Windows Phone documenting their origin (if known), mode of distribution, vulnerabilities being exploited, and its capabilities.}
	\item{Generating a template for the analysis report.}
\end{itemize}
\bibliographystyle{plain}
\bibliography{bibliography}
\end{document}
