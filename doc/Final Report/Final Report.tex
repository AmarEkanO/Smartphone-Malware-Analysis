\documentclass{llncs}

\begin{document}
\begin{center}
	
	% Upper part of the page. The '~' is needed because \\
	% only works if a paragraph has started.
	
	\textsc{\LARGE King's College London\\ \small School of Natural and Mathematical Sciences\\ \small Department of Informatics
	}\\[1.5cm]
	
	\textsc{\Large MSc Project Report}\\[0.5cm]
	
	% Title
	\hrule~\\[0.4cm]
	{ \huge \bfseries Smartphone Malware Analysis \\[0.4cm] }
	\hrule~\\[1.5cm]
	
	% Author and supervisor
	\noindent
	\begin{minipage}[t]{0.4\textwidth}
		\begin{flushleft} \large
			\emph{Author:}\\
			Amar Menezes\\(1435460)
		\end{flushleft}
	\end{minipage}%
	\begin{minipage}[t]{0.4\textwidth}
		\begin{flushright} \large
			\emph{Supervisor:} \\
			Dr.~Richard E. Overill
		\end{flushright}
	\end{minipage}
	
	\vfill
	
	% Bottom of the page
	{\large \today}
\end{center}


\begin{abstract} 
	 This report is surveys the two most popular smartphone platforms and the malware landscape associated with these platforms. This is the first part of the final report.
\end{abstract}
\section{Introduction} \label{Intro}
	Traditionally malware authors have targeted personal computers, workstations and servers. Mainly because they were potentially rich stores of sensitive information. Mobile phones on the other hand were mainly used for communication and did not offer much incentive for malware authors. However over the years smartphones and tablets have started replacing personal computers. People now use their smartphones not just to make calls and send messages, but for a host of other applications such as e-commerce transactions, data storage, navigation, socializing etc. Our smartphones now house a lot of our personal and financial information. This provides a huge incentive for malware authors to focus their efforts on attacking smartphones.

	In pre-smartphone era, where each mobile phone vendor used its own proprietary operating system and the phones capabilities itself were limited. Unlike the pre-smartphone era today's smartphones have far greater capabilities from their sophisticated hardware and their constantly improving platforms. The majority of smartphones today run on one of the three platforms which are Android\cite{android-home}, iOS\cite{iOS-home} and Windows Phone\cite{WindowsPhone-home}. As of the 1st quarter of 2015, Android holds a 79.4\% market share, followed by iOS with a 18.3\% and Windows Phone at 3.2\% share \cite{idc-smartphone-marketshare}.
	
	The remainder of this paper is structured as follows. Section \ref{Intro} introduces the problem of smartphone malware and its growth. Section \ref{malware-activities} summarises the malicious activities carried out by malware. Section \ref{classification} classifies the various malware types. Section \ref{platforms} summarises the architecture and security models of Android and iOS. Finally Section \ref{malware} describes the attack surface of each platform and malware families.
	
\subsection{Malware attack goals and distribution mechanisms} \label{malware-activities}
Suarez-Tangil et al.\cite{suarez2014evolution} categorised malware based on their attack goal and behaviour, method of distribution and privilege acquisition.\\\\
Attack goals and behaviour are summarised as
\begin{enumerate}
	\item{\textbf{Fraud:} Such as sending SMS/Calling premium numbers or holding device data or functionality ransom.}
	\item{\textbf{Sabotage:} Such as destroying data or rendering the device unusable.}
	\item{\textbf{Theft:} Exfiltration of user information (contact lists, messages, IMEI/IMSI numbers, call/location history etc) and/or user credentials (banking, social accounts, email, corporate accounts)}
	\item{\textbf{SPAM:} Agressive Adware}
	\item{\textbf{Service Misuse:} Such as snooping, spying or tracking of the user by exploiting device sensors. Another example is running a botnet without the users knowledge.}\\
\end{enumerate}
Methods of distribution and infection were categorized as follows
\begin{enumerate}
	\item{\textbf{Market to Device:} An attacker uses an app market to upload his/her malicious application. If markets are not policed for malicious content users are at risk of getting infected.}
	\item{\textbf{App to Device:} In this mode of distribution the attacker uses a vulnerable application to distribute his/her malicious application.}
	\item{\textbf{Web to Device:} This mode of distribution exploits vulnerabilities in web browsers to distribute malicious content.}
	\item{\textbf{SMS to Device:} Malware uses SMS/MMS to distribute malicious payloads. This was a popular strategy targeting the SymbianOS.}
	\item{\textbf{Network to Device:} This strategy exploits platform vulnerabilities or misconfigurations. Distribution uses either Device to Device (D2D) propagation or Cloud to Device (C2D) propagation.}
	\item{\textbf{USB to Device:} Malware infects devices when they are connected to an infected computer via a communication port usually USB.}\\
\end{enumerate}
Privilege acquisition is generally achieved via two methods
\begin{enumerate}
	\item{\textbf{User Manipulation:} An unsuspecting is tricked into granting privileges to malware. User manipulation is achieved via Social Engineering, use of repackaged applications from third-party sources, etc.}
	\item{\textbf{Technical Exploitation:} Here privileges are acquired by exploiting platform vulnerabilities or misconfigurations. Although vulnerabilities differ across platforms, most common attacks include API vulnerabilities, buffer overruns, injection attacks, protocol vulnerabilities etc.}
\end{enumerate}

\subsection{Malware capabilities} \label{capabilities}
Faruki et al. \cite{farukiandroid} categorised malware based on their capabilities within the context of smartphones.
\begin{itemize}
	\item{\textbf{Trojan:} Malicious apps that appear to have a benign purpose to the user, while performing harmful activities without the user being aware. Trojans are typically used in the exfiltration of sensitive data such as user credentials, contacts, messages etc. SMS Trojan families send SMS's to premium rate numbers without the user being aware.}
	\item{\textbf{Backdoors:} This type of malware infects systems exploiting platform weaknesses. Backdoors typically use root exploits to escalate privileges and evade detection.}
	\item{\textbf{Worm:} Malicious apps that create copies of itself which it distributes to other systems via networks and/or removable media.}
	\item{\textbf{Botnets:} These apps compromise the device to create a Bot, which forms part of a network of other such bots called a botnet. Bots are controlled by a Command and Control server and are used for malicious activities ranging from data exfiltration to denial of service attacks.}
	\item{\textbf{Spyware:} These apps perform malicious activites such as monitoring calls, contacts, messages, location, etc. It can also send this data to a remote server controlled by the attacker.}
	\item{\textbf{Adware:} These apps spam the user with unsolicited advertisements and notifications. These can create shortcuts on the home screen, steal bookmarks, and impair effective usage of the device.}
	\item{\textbf{Randsomware:} This type of malware locks the user out of his/her data and demands a ransom to unlock the data.}
\end{itemize}


\section{Smartphone Platforms} \label{platforms}
\subsection{Android}
\begin{itemize}
 \item{Origin (Creation, maintenance, current state of the project)}
 \item{Ecosystem (Smartphone market share, version(s) under review, software distribution repositories)}
 \item{Architecture/Software Stack}
 \item{Security Model}
 \item{Security features}
 \item{Android application execution}
\end{itemize}

\subsection{iOS}
\begin{itemize}
 \item{Origin}
 \item{Ecosystem}
 \item{Architecture/Software Stack}
 \item{Security Model}
 \item{Security features}
 \item{iOS application execution}
\end{itemize}

\section{Smartphone malware} \label{malware}

\subsection{Android}
\begin{itemize}
\item{Attack vectors}
\item{Known malware families and their capabilities}
\end{itemize}
\subsection{iOS}
\begin{itemize}
\item{Attack vectors}
\item{Known malware families and their capabilities}
\end{itemize}

\bibliographystyle{plain}
\bibliography{bibliography}
\end{document}
