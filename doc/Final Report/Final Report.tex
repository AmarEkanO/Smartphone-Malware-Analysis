\documentclass{llncs}

\begin{document}
\begin{center}
	
	% Upper part of the page. The '~' is needed because \\
	% only works if a paragraph has started.
	
	\textsc{\LARGE King's College London\\ \small School of Natural and Mathematical Sciences\\ \small Department of Informatics
	}\\[1.5cm]
	
	\textsc{\Large MSc Project Report}\\[0.5cm]
	
	% Title
	\hrule~\\[0.4cm]
	{ \huge \bfseries Smartphone Malware Analysis \\[0.4cm] }
	\hrule~\\[1.5cm]
	
	% Author and supervisor
	\noindent
	\begin{minipage}[t]{0.4\textwidth}
		\begin{flushleft} \large
			\emph{Author:}\\
			Amar Menezes\\(1435460)
		\end{flushleft}
	\end{minipage}%
	\begin{minipage}[t]{0.4\textwidth}
		\begin{flushright} \large
			\emph{Supervisor:} \\
			Dr.~Richard E. Overill
		\end{flushright}
	\end{minipage}
	
	\vfill
	
	% Bottom of the page
	{\large \today}
\end{center}


\begin{abstract} 
	 This report is surveys the two most popular smartphone platforms and the malware landscape associated with these platforms. This is the first part of the final report.
\end{abstract}
\section{Introduction} \label{Intro}
	Traditionally malware authors have targeted personal computers, workstations and servers. Mainly because they were potentially rich stores of sensitive information. Mobile phones on the other hand were mainly used for communication and did not offer much incentive for malware authors. However over the years smartphones and tablets have started replacing personal computers. People now use their smartphones not just to make calls and send messages, but for a host of other applications such as e-commerce transactions, data storage, navigation, socializing etc. Our smartphones now house a lot of our personal and financial information. This provides a huge incentive for malware authors to focus their efforts on attacking smartphones.

	In pre-smartphone era, where each mobile phone vendor used its own proprietary operating system and the phones capabilities itself were limited. Unlike the pre-smartphone era today's smartphones have far greater capabilities from their sophisticated hardware and their constantly improving platforms. The majority of smartphones today run on one of the three platforms which are Android\cite{android-home}, iOS\cite{iOS-home} and Windows Phone\cite{WindowsPhone-home}. As of the 1st quarter of 2015, Android holds a 79.4\% market share, followed by iOS with a 18.3\% and Windows Phone at 3.2\% share \cite{idc-smartphone-marketshare}.
	
	The remainder of this paper is structured as follows. Section \ref{Intro} introduces the problem of smartphone malware and its growth. Section \ref{malware-activities} summarises the malicious activities carried out by malware. Section \ref{classification} classifies the various malware types. Section \ref{platforms} summarises the architecture and security models of Android and iOS. Finally Section \ref{malware} describes the attack surface of each platform and malware families.
	
\subsection{Crimes being committed by malware} \label{malware-activities}
\begin{itemize}
	\item{Sending SMS/Calling premium numbers.}
	\item{Exfiltration of user information (contact lists, messages, IMEI/IMSI numbers, call/location history etc)}
	\item{Exfiltration of user credentials (banking, social accounts, email, corporate accounts)}
	\item{DoS attacks}
	\item{Snooping, spying or tracking of the user by exploiting device sensors.}
	\item{Agressive Adware}
	\item{User device being made inaccessible until a random amount is paid through online payment service}
\end{itemize}

\subsection{Malware Classification} \label{classification}
\begin{itemize}
	\item{Trojan}
	\item{Backdoors (rootkits)}
	\item{Worms}
	\item{Botnets}
	\item{Spyware}
	\item{Adware}
	\item{Randsomware}
\end{itemize}


\section{Smartphone Platforms} \label{platforms}
\subsection{Android}
\begin{itemize}
 \item{Origin (Creation, maintenance, current state of the project)}
 \item{Ecosystem (Smartphone market share, version(s) under review, software distribution repositories)}
 \item{Architecture/Software Stack}
 \item{Security Model}
 \item{Security features}
 \item{Android application execution}
\end{itemize}

\subsection{iOS}
\begin{itemize}
 \item{Origin}
 \item{Ecosystem}
 \item{Architecture/Software Stack}
 \item{Security Model}
 \item{Security features}
 \item{iOS application execution}
\end{itemize}

\section{Smartphone malware} \label{malware}

\subsection{Android}
\begin{itemize}
\item{Attack vectors}
\item{Known malware families and their capabilities}
\end{itemize}
\subsection{iOS}
\begin{itemize}
\item{Attack vectors}
\item{Known malware families and their capabilities}
\end{itemize}

\bibliographystyle{plain}
\bibliography{bibliography}
\end{document}
