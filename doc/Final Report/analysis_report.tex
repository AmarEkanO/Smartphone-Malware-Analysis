\newpage
\section{Sample Report} \label{report}
Title: Analysis of Simplelocker Trojan \\
Investigator: Amar Menezes\\
Date: 28-August-2015\\

\subsection{Executive Summary}
The sample analysed is a variant of the Simplelocker trojan for Android. The sample masquerades as the car racing game 'NFS Hot Pursuit' to trick users into downloading and installing it. On execution it accuses the user of downloading child porn and demands a payment of 500 rubles to unlock the phone. The sample steals the device ID (IMEI) and sends it to a remote server. Unlike other variants of its family this sample does not encrypt the filesystem nor does it lock the user out. This sample however disrupts the normal operation of the device by constantly flashing the ransom note on the users screen.

\subsection{Identification}
\begin{enumerate}
\item{Filename: Sample1.apk}
\item{size: 1.6 MB (1,632,117 bytes)}
\item{type: Android application}
\item{Hashes: 
\begin{itemize}
	\item[]{md5: 214abc0e56ee1fc231e2442bdd701ece}
	\item[]{sha1: 549ce491f75301b179393fd3f3b27d2f811b36f6}
	\item[]{sha256: c0cb135eef45bb8e411d47904ce638531d53473729c7752dc43c6d55d5ed86f8}
\end{itemize}
}
\item{Signing Information:\\
	Subject\\
	DN: C=US, ST=California, L=Mountain View, O=Android, OU=Android, CN=Android, E=android@android.com\\
	C: US\\
	E: android@android.com\\
	CN: Android\\
	L: Mountain View\\
	O: Android\\
	S: California\\
	OU: Android\\
	Issuer\\
	DN: C=US, ST=California, L=Mountain View, O=Android, OU=Android, CN=Android, E=android@android.com\\
	C: US\\
	E: android@android.com\\
	CN: Android\\
	L: Mountain View\\
	O: Android\\
	S: California\\
	OU: Android\\
	}
\item{TrID: Android Package}
\item{Aliases:\\ 
	37.vir\\
	NFS hot pursuit.apk\\
	214abc0e56ee1fc231e2442bdd701ece\\
	}
\end{enumerate}

\subsection{Capabilities}
\begin{itemize}
	\item{Capable of accessing the Internet}
	\item{Capable of accessing device specific data}
	\item{Prevents device from sleeping}
	\item{Starts automatically on bootup}
\end{itemize}

\subsection{Dependencies}
None

\subsection{Static Analysis}

\begin{enumerate}
\item{Permissions requested:
	\begin{itemize}
		\item{android.permission.WAKE\_LOCK (prevent phone from sleeping)}
		\item{android.permission.RECEIVE\_BOOT\_COMPLETED (automatically start at boot)}
		\item{android.permission.READ\_PHONE\_STATE (read phone state and identity)}
		\item{android.permission.ACCESS\_NETWORK\_STATE (view network status)}
		\item{android.permission.INTERNET (full Internet access)}
	\end{itemize}	
}	
	
\item{Activities: 
	\begin{itemize}
		\item[]{com.emsoft.regular.AltarSost}
	\end{itemize}
}
\item{Services: 
	\begin{itemize}
		\item[]{com.emsoft.regular.AltarSost1}
	\end{itemize}	
}
\item{Receivers:
	\begin{itemize}
		\item[]{com.emsoft.regular.AltarSost3}
		\item[]{com.emsoft.regular.AltarSDSost}
	\end{itemize}	
}
\item{Content Providers: None}
\item{Execution points of entry: com.emsoft.regular.AltarSost}
\item{Code related observations:
	\begin{itemize}
		\item{The code uses encryption APIs}
	\end{itemize}
}
\item{File contents:
	\begin{itemize}
		 \item[]{AndroidManifest.xml}
		 \item[]{META-INF/CERT.RSA}
		 \item[]{META-INF/CERT.SF}
		 \item[]{META-INF/MANIFEST.MF}
		 \item[]{classes.dex}
		 \item[]{res/drawable-hdpi/icon.png}
		 \item[]{res/drawable-ldpi/icon.png}
		 \item[]{res/drawable-mdpi/icon.png}
		 \item[]{res/drawable-xhdpi/icon.png}
		 \item[]{res/drawable/icon.png}
		 \item[]{res/layout/main.xml}
		 \item[]{res/raw/file}
		 \item[]{resources.arsc}
	\end{itemize}	
}
\end{enumerate}

\subsection{Dynamic Analysis}
\begin{enumerate}
\item{DNS Queries: ebuha4a.net}
\item{HTTP Conversations:\\ 
GET http://ebuha4a.net/keys/538043 HTTP/1.1\\
Host: ebuha4a.net\\
Connection: Keep-Alive\\
User-Agent: Apache-HttpClient/UNAVAILABLE (java 1.4)}
\item{Services/Processes started: com.emsoft.regular.AltarSost1}
\item{Data leaked: IMEI number}
\end{enumerate}

\subsection{Supporting Data}
\begin{enumerate}
\item{Network traces: capture.pcap}
\item{Screenshots: screenshot.jpg}
\end{enumerate}

\subsection{Conclusion}
This malware removal can be achieved by simply uninstalling it, as it has no persistence mechanism.
