\documentclass{llncs}
\usepackage{graphicx}
\usepackage{csquotes}


\begin{document}
\begin{center}
	
	% Upper part of the page. The '~' is needed because \\
	% only works if a paragraph has started.
	
	\textsc{\LARGE King's College London\\ \small School of Natural and Mathematical Sciences\\ \small Department of Informatics
	}\\[1.5cm]
	
	\textsc{\Large MSc Preliminary Project Report}\\[0.5cm]
	
	% Title
	\hrule\\[0.4cm]
	{ \huge \bfseries Smartphone Malware Analysis \\[0.4cm] }
	\hrule\\[1.5cm]
	
	% Author and supervisor
	\noindent
	\begin{minipage}[t]{0.4\textwidth}
		\begin{flushleft} \large
			\emph{Author:}\\
			Amar Menezes\\(1435460)
		\end{flushleft}
	\end{minipage}%
	\begin{minipage}[t]{0.4\textwidth}
		\begin{flushright} \large
			\emph{Supervisor:} \\
			Dr.~Richard E. Overill
		\end{flushright}
	\end{minipage}
	
	\vfill
	
	% Bottom of the page
	{\large \today}
	
\end{center}


\begin{abstract} 
	 This project aims to develop a process for malware analysis on an Android, iOS or Windows phone. This process would allow law enforcement, incident response teams and security researchers to efficiently analyse a given smartphone for malware and provide a technical summary of its capabilities, origins, the system weaknesses that it exploits and potential criminal activities.
\end{abstract}
\section{Introduction}
\begin{itemize}
	\item{\textbf{Why is there a need for this process?\\}}
	Over the last decade there has been a significant shift in computing from Desktops to mobile computing devices. Of these devices smartphones and tablets take up a large part of the market share. The growing popularity of these devices have caught the attention of malware writers. As everyday users rely on their devices for activities such as e-commerce, communication, navigation etc. There is a huge incentive for cyber criminals to exploit such devices that have weak security in place.
	\item{\textbf{Who benefits from the process?\\}}
	Malware is non-geographic, and can be orchestrated from any part of the world. Malware is a growing area as criminals look for financial gain with minimal risk. A robust set of processes to deal with malware on Smartphones will ensure that any criminal activity is detected and identified.
	\item{\textbf{Where and How will this process be used?\\}}
	This process can be used by law enforcement or incident response teams to assist in the reconstruction of events by malware and to gather forensic evidence to be used in legal prosecution. This process can also be used by security researchers in discovery and analysis of malicious software. The aim of the project is to develop a process for malware analysis on Android, iOS or Windows smartphones. This process would aid in e-discovery by identifying malware, its method of distribution and propagation, the security weaknesses that it exploits and the malicious activities it perpetrates.
\end{itemize}
\section{Related Work}
\begin{itemize}
	\item{What has currently been done in the field?}
	\item{What has worked?}
	\item{What has not worked?}
\end{itemize}
\section{Requirements}
\begin{itemize}
		\item{\textbf{Project objectives?\\}}
		\item{\textbf{Scope of the project?\\}}
\end{itemize}
\section{Timeline}
\begin{itemize}
		\item{How will the objectives be achieved?}
		\item{When will the project be completed?}
\end{itemize}

\bibliographystyle{plain}
\bibliography{bibliography}
\end{document}
