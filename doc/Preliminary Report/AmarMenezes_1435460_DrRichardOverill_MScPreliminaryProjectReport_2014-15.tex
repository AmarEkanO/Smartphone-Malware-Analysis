\documentclass{llncs}
\usepackage{gantt}

\begin{document}
\begin{center}
	
	% Upper part of the page. The '~' is needed because \\
	% only works if a paragraph has started.
	
	\textsc{\LARGE King's College London\\ \small School of Natural and Mathematical Sciences\\ \small Department of Informatics
	}\\[1.5cm]
	
	\textsc{\Large MSc Preliminary Project Report}\\[0.5cm]
	
	% Title
	\hrule~\\[0.4cm]
	{ \huge \bfseries Smartphone Malware Analysis \\[0.4cm] }
	\hrule~\\[1.5cm]
	
	% Author and supervisor
	\noindent
	\begin{minipage}[t]{0.4\textwidth}
		\begin{flushleft} \large
			\emph{Author:}\\
			Amar Menezes\\(1435460)
		\end{flushleft}
	\end{minipage}%
	\begin{minipage}[t]{0.4\textwidth}
		\begin{flushright} \large
			\emph{Supervisor:} \\
			Dr.~Richard E. Overill
		\end{flushright}
	\end{minipage}
	
	\vfill
	
	% Bottom of the page
	{\large \today}
\end{center}


\begin{abstract} 
	 This project aims to develop a process to detect, analyse and document malware on smartphones running either Android or iOS. This process would allow law enforcement, incident response teams and security researchers to efficiently analyse a given smartphone for malware and provide a technical summary of its capabilities, origins, the system weaknesses that it exploits and criminal activities being perpetrated.
\end{abstract}
\section{Introduction}
	Over the last decade there has been a significant shift in computing from Desktops to mobile computing devices. Of these devices smartphones and tablets take up a large part of the market share \cite{smartphone-market}. The growing popularity of these devices have caught the attention of malware writers. People have come to rely on their mobile devices for activities such as e-commerce, communication, navigation etc. This growing trend is something that is being actively exploited by cyber criminals \cite{kaspersky-cybercrime} \cite{symantec-cybercrime}. Security evaluations have revealed the ease and extent to which malware can be developed for different smartphone platforms \cite{mylonas2011smartphone} \cite{schmidt2010smartphone} \cite{grimes2012apple}. 
	When perpetrators of computer assisted crimes such as fraud, email \& Internet abuse, acquisition and storage of child pornography, software piracy, data theft etc are taken to court. The defence usually use the Trojan Horse Defence \cite{brenner2004trojan} to get an aquittal in such crimes. Having a clearly defined process to analyse malware is indispensable in such cases. 
	This process can be used by law enforcement or incident response teams to assist in the reconstruction of events by malware and to gather forensic evidence to be used in legal prosecution. This process can also be used by security researchers in discovery and analysis of malicious software. The aim of the project is to develop a process for malware analysis primarily on Android but can also be extended to iOS, Windows and other smartphone operating systems. This process would aid in e-discovery by identifying malware, its method of distribution and propagation, the security weaknesses that it exploits and the malicious activities it perpetrates.
\section{Related Work}

	Malware analysis for desktops and laptops has been well established. There is a wealth of tools for both static and dynamic analysis. Sophisticated methods for automated analysis have also been proposed and implemented \cite{willems2007toward} \cite{xie2013ipanda} as well as developing a technical vocabulary for malware analysis \cite{kirillov2011malware}.
	Smartphone malware analysis however is still in its infancy. Security researchers have made significant progress in trying to study smartphone malware \cite{zhou2012dissecting} \cite{suarez2014evolution} \cite{amamra2012smartphone} \cite{wu2012droidmat} \cite{zheng2013droid} \cite{blasing2010android} but most of this work has been on Android malware and very little on iOS or Windows. Primarily due to the fact that Android is the most popular operating system for smartphones as compared to iOS or Windows Phone. Also Apple and Microsoft review each app submitted to their app markets reducing the risk of malware getting through. Over the last five years tools to analyse android malware have matured, both commercial and open source. A few have been developed for iOS and other platforms but this is likely to change as these platforms increase their market share.
	Of all the proposed and implemented tools, none are capable of detecting all variations of malware and it is unlikely that such a tool will be available in the near future. However what we need is a clear and consistent methodology to analyse smartphone malware not just for technical analyst but also for law enforcement. Very little research has been done toward this end and this project is a step towards formalizing such a process.
\section{Requirements}
\subsection{Objectives}
The primary objectives of the project are as follows
\begin{itemize}
		\item{To research and summarize the malware landscape on Android and iOS.}
		\item{Develop a process to detect, analyse and document smartphone malware for e-discovery.}
		\item{The process should adhere to the four principles of computer-based electronic evidence.\cite{acpo-guidlines}}
		\item{The process should be tool and platform agnostic.}
\end{itemize}
The secondary objectives of the project are
\begin{itemize}
	\item{Automate the process or parts of the process to reduce manual effort and speed up the investigation.}
	\item{Build a catalogue of findings from the process that can be used for future investigations.}
	\item{Assemble a toolkit to aid the process.}
\end{itemize}
\subsection{Scope of the project}
Anti malware strategies and malware removal are not part of the project.

\section{Specification}
\begin{itemize}
	\item{\textbf{Platforms:}}
	The process would initially be designed for the analysis of Android malware but can also be extended to analyze iOS and Windows phone malware.
	\item{\textbf{Operating systems:}}
	For Android we cover versions 4.1.x (Jelly Bean) through 5.0 (Lollipop) and for iOS we cover 8.x.
	\item{\textbf{Methods of analysis:}}
	Analysis would involve both dynamic and static analysis.
	\item{\textbf{Reporting:}}
	The process would allow the generation of a report containing the technical details of the malware, its origin, its capabilities and how it is being used to commit crime. It would also identify the security weakness that facilitated infection.
\end{itemize}

\section{Timeline}
\begin{gantt}{12}{12}
	\begin{ganttitle}
		\titleelement{June 2015}{4}
		\titleelement{July 2015}{4}
		\titleelement{August 2015}{4}
	\end{ganttitle}
	\begin{ganttitle}
		\titleelement{Weeks}{4}
		\titleelement{Weeks}{4}
		\titleelement{Weeks}{4}
	\end{ganttitle}
	\begin{ganttitle}
		\numtitle{1}{1}{4}{1}
		\numtitle{1}{1}{4}{1}
		\numtitle{1}{1}{4}{1}
	\end{ganttitle}
	\ganttbar{Research Android and iOS}{0}{2}
	\ganttbarcon{a consecutive task}{2}{4}
\end{gantt}
\bibliographystyle{plain}
\bibliography{bibliography}
\end{document}
