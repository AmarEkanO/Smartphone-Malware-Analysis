\documentclass{llncs}
\usepackage{graphicx}
\usepackage{csquotes}
\usepackage{pgfgantt}


\begin{document}
\begin{center}
	
	% Upper part of the page. The '~' is needed because \\
	% only works if a paragraph has started.
	
	\textsc{\LARGE King's College London\\ \small School of Natural and Mathematical Sciences\\ \small Department of Informatics
	}\\[1.5cm]
	
	\textsc{\Large MSc Preliminary Project Report}\\[0.5cm]
	
	% Title
	\hrule~\\[0.4cm]
	{ \huge \bfseries Smartphone Malware Analysis \\[0.4cm] }
	\hrule~\\[1.5cm]
	
	% Author and supervisor
	\noindent
	\begin{minipage}[t]{0.4\textwidth}
		\begin{flushleft} \large
			\emph{Author:}\\
			Amar Menezes\\(1435460)
		\end{flushleft}
	\end{minipage}%
	\begin{minipage}[t]{0.4\textwidth}
		\begin{flushright} \large
			\emph{Supervisor:} \\
			Dr.~Richard E. Overill
		\end{flushright}
	\end{minipage}
	
	\vfill
	
	% Bottom of the page
	{\large \today}
\end{center}


\begin{abstract} 
	 This project aims to develop a process to detect, analyse and document malware on smartphones running either Android or iOS. This process would allow law enforcement, incident response teams and security researchers to efficiently analyse a given smartphone for malware and provide a technical summary of its capabilities, origins, the system weaknesses that it exploits and criminal activities being perpetrated.
\end{abstract}
\section{Introduction}
\begin{itemize}
	\item{\textbf{Why is there a need for this process?\\}}
	Over the last decade there has been a significant shift in computing from Desktops to mobile computing devices. Of these devices smartphones and tablets take up a large part of the market share \cite{smartphone-market}. The growing popularity of these devices have caught the attention of malware writers. People have come to rely on their mobile devices for activities such as e-commerce, communication, navigation etc. This growing trend is something that is being actively exploited by cyber criminals \cite{kaspersky-cybercrime} \cite{symantec-cybercrime} and security evaluations have revealed the ease and extent to which malware can be developed for different smartphone platforms \cite{mylonas2011smartphone} \cite{schmidt2010smartphone} \cite{grimes2012apple}. 
	\item{\textbf{Who benefits from the process?\\}}
	When perpetrators of computer assisted crimes such as fraud, email \& Internet abuse, acquisition and storage of child pornography, software piracy, data theft etc are taken to court. The defence usually use the Trojan Horse Defence \cite{brenner2004trojan} to get an aquittal in such crimes. Having a clearly defined process to analyse malware is indispensable in such cases. 
	\item{\textbf{Where and How will this process be used?\\}}
	This process can be used by law enforcement or incident response teams to assist in the reconstruction of events by malware and to gather forensic evidence to be used in legal prosecution. This process can also be used by security researchers in discovery and analysis of malicious software. The aim of the project is to develop a process for malware analysis primarily on Android but can also be extended to iOS, Windows and other smartphone operating systems. This process would aid in e-discovery by identifying malware, its method of distribution and propagation, the security weaknesses that it exploits and the malicious activities it perpetrates.
\end{itemize}
\section{Related Work}
\begin{itemize}
	\item{\textbf{What has currently been done in the field?}}
	Researchers have attempted to generate surveys on existing smartphone malware since 2007. Most of this work has been on classifying Android malware and very little on iOS or Windows. This is partly due to the reason that Apple and Microsoft review each app submitted to their app markets so there is a slim chance that malware can slip through. However this has occured in some occasions in the past. Jailbroken iPhones allow installation of apps from third party sources. This is been the most popular method of malware distribution on jailbroken iPhones. 
	\item{\textbf{What has worked?}}
	\item{\textbf{What has not worked?}}
	\item{\textbf{Why the need for yet another process?}}
\end{itemize}

\section{Requirements}
\subsection{Objectives}
The primary objectives of the project are as follows
\begin{itemize}
		\item{To research and summarize the malware landscape on Android and iOS.}
		\item{Develop a process to detect, analyse and document smartphone malware for e-discovery.}
		\item{The process should adhere to the four principles of computer-based electronic evidence.\cite{acpo-guidlines}}
		\item{To assemble a toolkit to aid the process.}
\end{itemize}
The secondary objectives of the project are
\begin{itemize}
	\item{Automate the process or parts of the process to reduce manual effort and speed up the investigation.}
	\item{Build a database of findings from the process that can be used for future investigations.}
\end{itemize}
\subsection{Scope of the project}

\section{Specification}
\begin{itemize}
	\item{Smartphone platforms covered?}
	The process would primarily be designed for the analysis of Android malware but can also be extended to analyze iOS and Windows phone malware.
	\item{Operating systems covered for each platform?}
	For Android we cover versions 4.1.x (Jelly Bean) through 5.0 (Lollipop) and for iOS we cover 8.x.
	\item{Methods of detection and analysis}
	Analysis would involve both dynamic and static analysis.
	\item{Reporting}
	The process would allow the generation of a report containing the technical details of the malware, its origin, its capabilities and how it is being used to commit crime. It would also identify the security weakness that facilitated infection.
\end{itemize}

\section{Timeline}
\begin{itemize}
		\item{How will the objectives be achieved?}
		\item{When will the project be completed?}
\end{itemize}
	\begin{ganttchart}{1}{12}
		\gantttitle{Project Roadmap}{12} \\
		\gantttitlelist{1,...,12}{1} \\
		\ganttgroup{Group 1}{1}{7} \\
		\ganttbar{Task 1}{1}{2} \\
		\ganttlinkedbar{Task 2}{3}{7}
		\ganttnewline
		\ganttmilestone{Milestone}{7}
		\ganttnewline
		\ganttbar{Final Task}{8}{12}
		\ganttlink{elem2}{elem3}
		\ganttlink{elem3}{elem4}
	\end{ganttchart}
\bibliographystyle{plain}
\bibliography{bibliography}
\end{document}
